\section{Neuronale Netzwerke}

\subsection{Geschichte}

Im Jahr 1943 wurde die erste Arbeit darüber geschrieben, wie Neuronen im Gehirn funktionieren könnten und die Autoren Warren McCulloch und Walter Pitts experimentierten sogar damit diese mit elektronischen Schaltkreisen nachzubauen.\footnote[6]{\cite[]{alogicalcalculus}} In den 1950er Jahren haben Forscher von IBM daran gearbeitet ein NN\footnote[7]{Kurzform für Neuronales Netzwerk, wird ab jetzt weiterhin verwendet.} mit einem Computer zu simulieren.\footnote[8]{\cite[Absatz 3]{nnhistory}} 

\subsubsection{Zeitstrahl}

\begin{chronology}[10]{1940}{2020}{\textwidth}
    \event{1943}{Erste Arbeit und Experimente}
    \event[1950]{1960}{Bemühungen, ein NN digital umzusetzen}
\end{chronology}
    
\subsection{Aufbau}

\subsection{Funktionsweise}

\subsubsection{Trainieren - Backpropagation}