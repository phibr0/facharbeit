\section{Einführung}
Im Enterprise und Forschungsbereich spielt Machine Learning schon seit vielen Jahren eine bedeutende Rolle. Doch wie kann es dem Endnutzer, zum Beispiel in mobilen Apps, weiterhelfen?

\subsection{Wahl des Themas}

Seitdem ich im Jahr 2017 meinen ersten richtigen Kontakt mit der Programmierung von Microcontrollern (Arduinos) hatte, habe ich mich stark für die Entwicklung von Software interessiert. Dies war ein guter Einstieg, da man dort schnell und recht einfach Ergebnisse, wie zum Beispiel eine blinkende LED, erzielt.

Auch habe ich mich seitdem immer für die "`neuen großen Technologien"' wie Blockchain oder Machine Learning interessiert. Zum Thema Machine Learning habe ich zuvor noch nicht viel gemacht, daher ergriff ich die Chance dieses Jahr meine Facharbeit über dieses Thema zu schreiben.


\begin{quotation}
    AI is profound, and we are at a point—and it will get better and better over time—where the GPU is getting so powerful there’s so much capability to do unbelievable things. What all of us have to do is to make sure we are using AI in a way that is for the benefit of humanity, not to the detriment of humanity.\footnote{\cite[Tim Cook (CEO von Apple) In einem Interview mit MIT Technology Review]{timcookquote}}
\end{quotation}

Ich persönlich finde dieses Zitat sehr wichtig; es ist jetzt über 3 Jahre alt und bis heute hat sich enorm viel in diesem Bereich getan. Wir haben nun GPU's, welche speziell auf mathematische Berechnungen mit Tensoren optimiert sind und so das Trainieren von Neuronalen Netzen um ein Vielfaches beschleunigen.\footnote{\cite[NVIDIA Grafikprozessoren mit integrierten Tensor Kernen]{nvidiatensorcores}}

Des weiteren ist es mir, genauso wie Cook, wichtig, diese mächtige Technologie nicht zu missbrauchen\footnote{Beispiel: Autonome Waffen, wie Drohnen, welche Ziele autonom erfassen können oder auch "`Deep-fakes"'}, sondern gute Dinge mit ihr zu schaffen: wie beispielsweise im Bereich der Medizin. In diesem Bereich wurden schon viele beachtliche Anwendungszwecke gefunden, so hat Google's Tochterfirma DeepMind im Dezember 2020 eine Technologie\footnote{Künstliche Intelligenz} präsentiert, welche das Falten von Proteinen akkurat prognostizieren kann; dies war vorher nur sehr langsam und deutlich ungenauer möglich.\footnote{\cite{deepmindprotein}}

\subsection{Ziel der Arbeit}

Mein persönliches Ziel ist es, mehr über den Aufbau von Neuronalen Netzen und die Funktionsweise von Machine Learning zu lernen. Außerdem möchte ich auch ein praktisches Ergebniss haben, dafür habe ich im Kapitel \ref{labelcheck} eine App entwickelt, welche dem Nutzer mehr Informationen über Produkte beim einkaufen liefern soll.